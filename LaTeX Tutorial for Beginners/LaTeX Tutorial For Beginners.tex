% !TEX TS-program = pdflatexmk
\documentclass{article}

\usepackage{amsmath}
\usepackage{tabularx}
\usepackage{hyperref}

\hypersetup{
	colorlinks = true,
	linktoc = all,
	linkcolor = blue,
	urlcolor = blue
}

\title{\LaTeX{} Tutorial for Beginners}
\author{Shane Miller}
\date{July 16th, 2019}

\tolerance = 1
\hyphenpenalty = 10000

\begin{document}
	\pagenumbering{gobble}
	\maketitle
	\newpage
	
	\tableofcontents
	\newpage

	\pagenumbering{arabic}
	
	\paragraph{Important Note:}
		A few of the examples in this document are specific to the article documentclass. For example, in the article documentclass, sectioning starts at 1. In other things such as report, it starts at 0.1. Due to this, I would suggest starting your first project using the article document and then experimenting with other documentclasses once you are comfortable with \LaTeX{}. When you eventually switch to trying out different documentclasses, don\rq{}t take everything here as exact. More or fewer things may be available for that documentclass. For example, I give an example of three commands you can use, but in other documentclasses more or fewer commands may be available to use. $\backslash$subtitle is available in the book documentclass, but not the article documentclass. In the intermediate tutorial I will go over the use of a different package that is commonly used instead of the default documentclass options \LaTeX{} provides. Please take these things into consideration when using this document for reference.
	
	\section{Installing \LaTeX{}}
		\subsection{Windows Instructions}
			\begin{enumerate}
				\item Go to \href{https://miktex.org/}{\textbf{miktex.org}}.
				\item Navigate to \href{https://miktex.org/download}{\textbf{Downloads}}.
				\item Download the MiKTeX installer for Windows.
				\item Run the MiKTeX installer once it finishes downloading.
				\item Agree and continue.
				\item Set preferred paper size and set \lq\lq{}Install Missing Packages\rq\rq{} to automatic.
				\item When this is finished you will have the MiKTeX Console and the TeXworks editor. Make sure you have both.
				\item Run the MiKTeX Console.
				\item Select \lq\lq{}Check for Updates\rq\rq{} and allow it to install any it finds.
				\item Once updates are finished installing, close the MiKTeX Console and open TeXworks.
				\item Begin writing your first \LaTeX{} document.
				\item When saving, I recommend creating a LaTeX folder in which to store all of your \LaTeX{} documents and then creating a subfolder for that specific document (eg: LaTeXDocuments \(\rightarrow\) MyLaTeXProject \(\rightarrow\) project files). When you save your .tex file, it will save a couple others there as well as the PDF when compiled.
			\end{enumerate}

		\subsection{Mac Instructions}
			\begin{enumerate}
				\item Go to the MacTeX website \href{http://www.tug.org/mactex/}{\textbf{here}}.
				\item Click \href{http://www.tug.org/mactex/mactex-download.html}{\textbf{MacTeX Download}}.
				\item Click on \href{http://tug.org/cgi-bin/mactex-download/MacTeX.pkg}{\textbf{MacTeX.pkg}} to download MacTeX.
				\item MacTeX.pkg will appear in your downloads folder once it is finished downloading. Double-click it to run the installer.
				\item Follow the instructions in the installer to install MacTeX.
				\item Once complete. navigate  in finder to Applications.
				\item In Applications, open the folder named \lq\lq{}TeX.\rq\rq{}
				\item Inside it you will see an application named \lq\lq{}TeX Shop.\rq\rq{} Open it.
				\item Begin writing your first \LaTeX{} document.
				\item When saving, I recommend creating a LaTeX folder in which to store all of your  \LaTeX{} documents and then creating a subfolder for that specific document (eg: LaTeXDocuments \(\rightarrow\) MyLaTeXProject \(\rightarrow\) project files). When you save your .tex file, it will save a couple others there as well as the PDF when compiled.
			\end{enumerate}

		\subsection{Linux Instructions}
			\begin{enumerate}
				\item In the command line type: \texttt{sudo apt-get install texlive-full}.
				\item In the command line type: \texttt{sudo apt-get install texmaker}.
				\item In the command line type: \texttt{texmaker}.
				\item \(\text{The UI for TeXmaker should have booted. From there select: File} \rightarrow \text{New.}\)
				\item Begin writing your first \LaTeX{} document.
				\item When saving, I recommend creating a LaTeX folder in which to store all of your \LaTeX{} documents and then creating a subfolder for that specific document (eg: LaTeXDocuments \(\rightarrow\) MyLaTeXProject \(\rightarrow\) project files). When you save your .tex file, it will save a couple others there as well as the PDF when compiled.
			\end{enumerate}

	\section{\LaTeX{} Important Notes}
		\begin{itemize}
			\item A \LaTeX{} file might need to be compiled more than once.
			\begin{itemize}
				\item This can happen for a couple of reasons, the main one being if something you use relies on something further down in the code.
				\item The most common example of this is the title page. You will make sections after you use $\backslash$maketitle to have \LaTeX{} automatically make your title page. Because of this, your \LaTeX{} will need to be compiled once to compile the rest of the document, and then one more time for your title page to add the sections to itself.
				\item On Mac, if you add \texttt{\% !TEX TS-program = pdflatexmk} to the beginning of your \LaTeX{} document, that may compile it the correct number of times automatically if you have latexmk installed (which is often automatically installed with the distribution of \LaTeX{} I included above).
				\item On Linux, in TeXmaker, navigate to Options \(\rightarrow\) Configure TeXmaker \(\rightarrow\) Quick Build and add \texttt{latexmk -pdf \%.tex} as a quick build command. If you need to for something else in the future, you can edit this command or make a different one.
			\end{itemize}
			\item \LaTeX{} is very powerful. It can do much more than make a simple text-based document. You can add images, hyperlinks, flow charts, musical scores, circuit diagrams, and much more. I will \textbf{not} cover everything \LaTeX{} can do in this three-part tutorial series. I am only covering some of the more important basics and a select few of the more advanced features. If there is something you would like to do and it is not included in this tutorial series, please look it up online; there is often a package that allows you to do whatever you need doing.
			\begin{itemize}
				\item A good source to look for answers is on the \TeX{} StackExchange \href{https://tex.stackexchange.com/}{\textbf{here}}.
			\end{itemize}
			\item After compiling your .tex file, some auxillary files will be added into the location your .tex file is. These auxillary files are simply used by \LaTeX{} to help compile. You can ignore these files and only need to include the .tex when sending it to someone. It will also create a PDF file in the same location. It is up to you whether or not you want to include this when sending someone the .tex file.
		\end{itemize}
		
	\section{General Formatting of a \LaTeX{} Document}
		\begin{itemize}
			\item When writing in \LaTeX{}, indentation does not matter, but if you open the .tex file associated with this PDF, you can see the indentation convention I follow.
			\begin{itemize}
				\item I indent based on \lq\lq{}level.\rq\rq{} This means that each time I go a level deeper (eg: section to subsection), I add one more indent.
				\item Additionally, I add an empty line between each unrelated piece of the document.
			\end{itemize}
			\item A \LaTeX{} document\rq{}s first line should always be a declaration of the document class.
			\begin{itemize}
				\item This is done by typing $\backslash$documentclass\{x\} where x is the type of document you are making.
				\item The different document classes and their uses are as follows:
			\end{itemize}
			\def\arraystretch{1.5}
			\begin{tabularx}{\textwidth}{|l|X|}
				\hline
				article & For articles in scientific journals, presentations, short reports, program documentation, invitations, et cetera.\\
				\hline
				IEEEtran & For articles with the IEEE Transactions format.\\
				\hline
				proc & A class for proceedings based on the article class.\\
				\hline
				report & For longer reports containing several chapters, small books, thesis, et cetera.\\
				\hline
				book & For full books.\\
				\hline
				slides & For slides. The class uses big sans serif letters.\\
				\hline
				memoir & For changing sensibly the output of the document. It is based on the book class, but you can create any kind of document with it.\\
				\hline
				letter & For writing letters.\\
				\hline
				beamer & For writing presentations.\\
				\hline
			\end{tabularx}
			\begin{itemize}
				\item For more information please look \href{https://en.wikibooks.org/wiki/LaTeX/Document_Structure#Document_classes}{\textbf{here}}.
			\end{itemize}
			\item When you want to begin actually writing your document you use $\backslash$begin\{document\}.
			\begin{itemize}
				\item This allows you to start writing out text and using other commands shown later in this tutorial (such as $\backslash$section).
			\end{itemize}
			\item The area between $\backslash$documentclass and $\backslash$begin\{document\} is called the preamble.
			\begin{itemize}
				\item This is the area where you set up the various things you may need for your document. You can use things like $\backslash$usepackage, $\backslash$author, and other things. I will cover these in more detail later.
			\end{itemize}
			\item Anything that uses $\backslash$begin\{x\} needs a $\backslash$end\{x\} after all of the content inside that block.
			\item When trying to use reserved characters (characters reserved for commands in \LaTeX{}), you need to use $\backslash$ to escape them. (Ex: To type \$ you need to type$\backslash$\$)
			\begin{itemize}
				\item Reserved characters include \$, \{, \}, \#, \%, $\backslash$, and many others.
				\item An easy way to tell if the character you are using is a reserved character is the fact that it will be a different color from the rest of your text.
				\item The one exception to the rule of using $\backslash$ to escape reserved characters is $\backslash$ itself. To type $\backslash$ in text you would use $\backslash$backslash.
			\end{itemize}
			\item To comment out a line in \LaTeX{} (so that line isn\rq{}t read at compile-time), preface the line with \%.
			\item To type a new-line character in \LaTeX{}, you use a $\backslash$$\backslash$. This is the equivalent of hitting return in various other word processors.
			\item If you wish to add text directly after a command that does not use \{\}, you need to encapsulate the command in \$\rq{}s.
			\begin{itemize}
				\item To type $\backslash$example in \LaTeX{} you need to encapsulate the $\backslash$backslash command in \$\rq{}s like this: \$$\backslash$backslash\$example
			\end{itemize}
		\end{itemize}
	
	\section{Title Page}
		\begin{itemize}
			\item To add a title page, simply add $\backslash$title\{Title Name Here\}, $\backslash$author\{Author Name Here\}, and $\backslash$date\{Date Here\} to the preamble.
			\begin{itemize}
				\item Note: If you would prefer to  have the date be updated at compile-time rather than you having to change the date each day you are working on a document, leave the $\backslash$date\{Date Here\} out of the preamble. In this case, the date will still be created and added to the title page, but it will automatically update what date it shows each time you compile your PDF.
			\end{itemize}
			\item After your $\backslash$begin\{document\} declaration add the following just below it:
			\begin{itemize}
				\item $\backslash$maketitle
				\item $\backslash$newpage
			\end{itemize}
			\item If you don\rq{}t want your title page numbered, add this command just above the $\backslash$maketitle declaration:
			\begin{itemize}
				\item $\backslash$pagenumbering\{gobble\}
			\end{itemize}
			\item You can also use $\backslash$pagenumbering to begin the numbering in arabic numbering or roman numeral numbering. This can be done with either of the following commands:
			\begin{itemize}
				\item $\backslash$pagenumbering\{arabic\}
				\item $\backslash$pagenumbering\{roman\}
			\end{itemize}
			\item If you call $\backslash$pagenumbering\{gobble\} on your title page, your document will not start numbering things again until you call $\backslash$pagenumbering\{arabic/roman\}.
		\end{itemize}
	
	\section{Table of Contents and Hyperlinking}
		\begin{itemize}
			\item To hyperlink in \LaTeX{}, you need the hyperref package.
			\begin{itemize}
				\item Get this by typing $\backslash$usepackage\{hyperref\} in your preamble.
			\end{itemize}
			\item To set up your hyperlinks to look nicer, add the following to your preamble after your $\backslash$usepackage declarations:
			\begin{verbatim}
				\hypersetup{
				      colorlinks = true,
				      urlcolor = blue
				}
			\end{verbatim}
			\begin{itemize}
				\item \texttt{colorlinks} colors your links rather than adding a box around them.
				\item \texttt{urlcolor} sets the color of URL links to whatever color you choose. Default is a reddish-pink and here I set it to be a more common blue.
			\end{itemize}
			\item If you plan on having a Table of Contents as well, use the following instead:
			\begin{verbatim}
				\hypersetup{
				      colorlinks = true,
				      linktoc = all,
				      linkcolor = blue,
				      urlcolor = blue
				}
			\end{verbatim}
			\begin{itemize}
				\item \texttt{linktoc} determines what is linked on your table of contents. Here I set it to all, and thus everything on the table of contents for this document is linked.
				\item \texttt{linkcolor} sets the color your table of contents links are. This color selection is separate from the URL hyperlink color set with \texttt{urlcolor}.
			\end{itemize}
			\item To add a Table of Contents to your document simply use these two commands after the $\backslash$newpage of your title page declaration:
			\begin{itemize}
				\item $\backslash$tableofcontents
				\item $\backslash$newpage
			\end{itemize}
		\end{itemize}
	
	\section{Section Types: Main Section}
		\begin{itemize}
			\item Made by using $\backslash$section.
			\item Sections change the number that comes before the first decimal and resets any subsections that come after it.
			\begin{itemize}
				\item If you were on section 1.2.6 and you called $\backslash$section, you would be on section 2.
			\end{itemize}
			\item Any text you want written in the section can go directly below your $\backslash$section declaration.
		\end{itemize}

		\subsection{Section Types: Subsection}
			\begin{itemize}
				\item Made by using $\backslash$subsection.
				\item Subsections change the number that comes after the first decimal and resets any subsections that come after it.
				\begin{itemize}
					\item If you were on section 1.2.6 and you called $\backslash$subsection, you would be on section 1.3.
				\end{itemize}
				\item Any text you want written in the subsection can go directly below your $\backslash$subsection declaration.
			\end{itemize}
	
			\subsubsection{Section Types: Subsubsection}
				\begin{itemize}
					\item Made by using $\backslash$subsubsection.
					\item Subsubsections change the number that comes after the second decimal.
					\begin{itemize}
						\item If you were on section 1.2.6 and you called $\backslash$subsubsection, you would be on section 1.2.7.
					\end{itemize}
					\item Any text you want written in the subsubsection can go directly below your $\backslash$subsubsection declaration.
					\item There is nothing lower than a subsubsection. A subsubsubsection does not exist.
				\end{itemize}

	\section{Paragraphs}
		\paragraph{Main Paragraph}
			\begin{itemize}
				\item Made by using $\backslash$paragraph.
				\item Any text you want written in the paragraph can go directly below your $\backslash$paragraph declaration.
			\end{itemize}
	
		\subparagraph{Subparagraph}
			\begin{itemize}
				\item Made by using $\backslash$subparagraph.
				\item Any text you want written in the subparagraph can go directly below your $\backslash$subparagraph declaration.
			\end{itemize}
	
	\section{Text Formatting}
		\subsection{Font Size}
			\begin{itemize}
				\item The following commands allow you to change the font size in \LaTeX{} relative to the default document font size:\\\\
				\def\arraystretch{2.25}
				\begin{tabularx}{\textwidth}{|l|X|X|}
					\hline
					Command & Example & Description\\
					\hline
					\{$\backslash$tiny Enter Text Here\} & {\tiny Example Text} & Makes the font size two units smaller.\\
					\hline
					\{$\backslash$small Enter Text Here\} & {\small Example Text} & Makes the font size one unit smaller.\\
					\hline
					\{$\backslash$large Enter Text Here\} & {\large Example Text} & Makes the font size one unit larger.\\
					\hline
					\{$\backslash$huge Enter Text Here\} & {\huge Example Text} & Makes the font size two units larger.\\
					\hline
				\end{tabularx}
			\end{itemize}

		\subsection{Font Style}
			\begin{itemize}
				\item The following commands allow you to use different font styles in \LaTeX{}:\\\\
				\def\arraystretch{1.75}
				\begin{tabularx}{\textwidth}{|l|X|X|}
					\hline
					Command & Example & Description\\
					\hline
					$\backslash$textbf\{Enter Text Here\} & \textbf{Example Text} & Bold\\
					\hline
					$\backslash$textit\{Enter Text Here\} & \textit{Example Text} & Italic\\
					\hline
					$\backslash$texttt\{Enter Text Here\} & \texttt{Example Text} & Typewriter\\
					\hline
					$\backslash$textrm\{Enter Text Here\} & \textrm{Example Text} & Serif (Roman)\\
					\hline
					$\backslash$underline\{ Enter Text Here\} & \underline{Example Text} & Underline\\
					\hline
				\end{tabularx}
			\end{itemize}
		
		\subsection{Verbatim}
			\begin{itemize}
				\item If you want to type a block of code or anything that may contain \LaTeX{} commands, you can use the verbatim environment using $\backslash$begin\{verbatim\} and $\backslash$end\{verbatim\}.
				\begin{itemize}
					\item The verbatim environment will write whatever is in it as text without running any commands that may be in it.
					\item The verbatim environment can\rq{}t recogonize tabs, but can recogonize spaces. If you need to use a tab, convert them into spaces. I\rq{}ve found that six spaces indents similarly to a tab in the \LaTeX{} editor.
					\item If you want to use the verbatim environment in-line, then you can use $\backslash$verb\textbar{}text here\textbar{}.
				\end{itemize}
			\end{itemize}

	\section{Lists}
		\subsection{Bulleted List}
			\begin{itemize}
				\item To make a bulleted list in \LaTeX{}, you need to use the \textit{itemize} declaration within $\backslash$begin\{\} and $\backslash$end\{\}.
				\begin{itemize}
					\item This will look like this: $\backslash$begin\{itemize\} and $\backslash$end\{itemize\}.
				\end{itemize}
				\item For each bullet point you need to use $\backslash$item followed by a space and then whatever it is you want after the bullet. This will go between the $\backslash$begin and $\backslash$end.
				\item If you want to make a sub-bullet point (the dashes in this document), within the $\backslash$begin\{itemize\} and $\backslash$end\{itemize\}, you need to add another $\backslash$begin\{itemize\} and $\backslash$end\{itemize\}.
				\begin{itemize}
					\item Between the second set of \textit{itemize}, you  use $\backslash$item followed by a space and then whatever it is you want after the bullet.
				\end{itemize}
			\end{itemize}

		\subsection{Ordered List}
			\begin{itemize}
				\item To make a numbered list in \LaTeX{}, you need to use the \textit{enumerate} declaration within $\backslash$begin\{\} and $\backslash$end\{\}.
				\begin{itemize}
					\item This will look like this: $\backslash$begin\{enumerate\} and $\backslash$end\{enumerate\}.
				\end{itemize}
				\item For each bullet point you need to use $\backslash$item followed by a space and then whatever it is you want after the bullet. This will go between the $\backslash$begin and $\backslash$end.
				\item If you want to make a sub-enumeration point (indenting and labeling with a, b, c, et cetera instead of 1, 2, 3, et cetera), within the $\backslash$begin\{enumerate\} and $\backslash$end\{enumerate\}, you need to add another $\backslash$begin\{enumerate\} and $\backslash$end\{enumerate\}.
				\begin{itemize}
					\item Between the second set of \textit{enumerate}, you  use $\backslash$item followed by a space and then whatever it is you want after the bullet.
				\end{itemize}
			\end{itemize}
	
		\subsection{Combined Lists}
			\begin{itemize}
				\item You can use both itemize and enumerate together. This will give you a mix of bulleted and numbered lists.
				\begin{itemize}
					\item For Example: If you have a numbered list, but want to include a few bullet points regarding one of the numbers, you can nest an itemized list in it if you don\rq{}t want to number those points in an order.
				\end{itemize}
				\item You can nest as many lists as you would like.
			\end{itemize}

	\section{Math}
		\begin{itemize}
			\item To beginyou should use the  package amsmath by calling $\backslash$usepackage\{amsmath\} in the preamble.
			\begin{itemize}
				\item While \LaTeX{} does have a way to type an equation without a package (using \$\$equation\$\$), it is highly discouraged as amsmath is better in every way. It uses the default \LaTeX{} math as a base, so there shouldn\rq{}t be anything in default \LaTeX{} math that you can\rq{}t do in amsmath.
			\end{itemize}
			\item When writing math, you want to use $\backslash$begin\{equation\}, your equation, and then $\backslash$end\{equation\}.
			\item Using this without a package will automatically number each line.
			\item To avoid this, you can use $\backslash$begin\{equation*\}, your equation, and then $\backslash$end\{equation\}.
		\end{itemize}
		
		Here is an example of f(x) = x\string^2/4\string^$\backslash$sigma as a \LaTeX{} equation:
		\\
		\begin{equation}
			f(x) = x^2/4^\sigma
		\end{equation}

		Here is an example of  f(x) = x\string^2/4\string^$\backslash$sigma as an amsmath equation:\\
		\begin{equation*}
			f(x) = x^2/4^\sigma
		\end{equation*}
		\\
		\begin{itemize}
			\item In \LaTeX{}, you can type a bunch of mathematical symbols using various commands. As seen above, $\backslash$sigma will get you \(\sigma\). Please see \href{http://tug.ctan.org/info/undergradmath/undergradmath.pdf}{\textbf{here}} or \href{http://tug.ctan.org/info/short-math-guide/short-math-guide.pdf}{\textbf{here}} for information relating to what commands can be used.
			\item To use an equation or any math command within a normal sentence (such as this one), use $\backslash$(equation here$\backslash$) wherever you want the equation in the sentence.
			\begin{itemize}
				\item The parenthesis escaped by the $\backslash$ will not show up in the equation. If you want any parenthesis in your equation, use non-escaped parenthesis inside the escaped parenthesis.
			\end{itemize}
			\item Here are a few example commands:
			\begin{itemize}
				\item You can use the command $\backslash$frac\{x\}\{y\} to make a fraction look like \(\frac{x}{y}\) instead of \(x/y\).
				\item You can use the command x $\backslash$times y  or x $\backslash$cdot y to make multiplication look like \(x \times y\) or \(x \cdot y\) instead of \(x*y\).
				\item $\backslash$sqrt\{x\} will make \(\sqrt{x}\).
				\item x\_\{y\} will make a subscript such as \(x_{y}\).
				\item x\string^y will make a superscript (or exponent) such as \(x^y\).
				\item $\backslash$text\{text here\} will allow you to type regular text within an equation block.
				\item Encapsulating your equation in $\backslash$boxed\{equation here\} will allow you to put a box around your equation.
			\end{itemize}
		\end{itemize}

	\section{Other Resources}
		Here is a list of other great resources regarding learning LaTeX. Most of these were used throughout making this tutorial and contain some information not included in this tutorial.
		\begin{itemize}
			\item \href{http://www.rpi.edu/dept/arc/docs/latex/latex-intro.pdf}{\textbf{RPI \LaTeX{} Intro}}
			\item \href{https://www.latex-tutorial.com/tutorials/}{\textbf{A Simple Guide to \LaTeX{}}}
			\item \href{https://en.wikibooks.org/wiki/LaTeX}{\textbf{\LaTeX{} Wikibook}}
			\item \href{http://www.docs.is.ed.ac.uk/skills/documents/3722/3722-2014.pdf}{\textbf{University of Edinburgh\rq{}s \LaTeX{} for Beginners}}
			\item \href{https://tex.stackexchange.com/}{\textbf{\TeX{} StackExchange}}
			\item \href{http://tug.ctan.org/info/short-math-guide/short-math-guide.pdf}{\textbf{Short Math Guide for \LaTeX{}}}
			\item \href{http://tug.ctan.org/info/undergradmath/undergradmath.pdf}{\textbf{\LaTeX{} Math Command Cheat Sheet}}
			\item The .tex file included in the zip for this document has everything that was used to write this tutorial. Please look through it to see how things in this document were used in practice.
		\end{itemize}
\end{document}