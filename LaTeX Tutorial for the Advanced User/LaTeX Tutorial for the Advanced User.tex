% !TEX TS-program = pdflatexmk
\documentclass{article}

\usepackage{hyperref}

\hypersetup{
	colorlinks = true,
	linktoc = all,
	linkcolor = blue,
	urlcolor = blue
}

\title{\LaTeX{} Tutorial for the Advanced User}
\author{Shane Miller}


\tolerance = 1
\hyphenpenalty = 10000

\begin{document}
	\pagenumbering{gobble}
	\maketitle
	\newpage


	\tableofcontents
	\newpage

	\pagenumbering{arabic}
	\section{Introduction}
	In my previous tutorial documents,\textit{\LaTeX{} for Beginners}, I demonstrated many basic functions \LaTeX{} supports. In this tutorial, I will go over many more of the more advanced uses of \LaTeX{} of which most are going to be for specific users. For example, in this document I will go over how to make curcuit diagrams in \LaTeX{}. That chapter will be very useful for people in the electrical or computer engineering field, but likely would be of no use to someone in a mostly pure mathematical field such as statistics. Please make use of the table of contents, skipping to sections that are relevant to you. If you are here simply to learn (similarly to me as some of these features are not necessarily relevant to my carrer nor hobbies) please feel free to go through the entire document.

	\section{Use of Foreign Languages}
	

	\section{Circuit Diagrams}

	
	\section{Musical Composition}

	
	\section{Flow Charts}


	\section{3-Dimensional Graphics}


	\section{Managing Large \LaTeX{} Documents}


	\section{Other Resources}
	
\end{document}